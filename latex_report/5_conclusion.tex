\section{Kết luận}

Bài tập lớn đã nghiên cứu và triển khai các phương pháp ước lượng kênh truyền vô tuyến, bao gồm Least Squares (LS), Minimum Mean Square Error (MMSE) và kỹ thuật học sâu (Deep Learning). Qua quá trình phân tích và mô phỏng, các kết quả cho thấy:

\begin{enumerate}
    \item \textbf{Phương pháp LS} đơn giản và dễ triển khai, nhưng độ chính xác bị giới hạn trong môi trường có tỷ lệ tín hiệu trên nhiễu thấp (SNR thấp). Tuy nhiên, đây vẫn là một giải pháp hiệu quả cho các hệ thống có tài nguyên tính toán hạn chế.
    
    \item \textbf{Phương pháp MMSE} đã cải thiện đáng kể độ chính xác bằng cách khai thác thông tin thống kê về kênh và nhiễu. Tuy nhiên, nó đòi hỏi khả năng tính toán cao và thông tin chính xác về kênh, điều này có thể trở thành thách thức trong thực tế.
    
    \item \textbf{Phương pháp học sâu} thể hiện tiềm năng lớn với khả năng xử lý hiệu quả trong các môi trường phức tạp. Kết quả mô phỏng cho thấy mạng nơ-ron tích chập (CNN) có thể đạt độ chính xác cao, đặc biệt khi số lượng pilot đủ lớn. Việc sử dụng kỹ thuật học sâu không chỉ cải thiện độ chính xác mà còn mở ra cơ hội áp dụng cho các hệ thống hiện đại như 5G và MIMO.
\end{enumerate}

Từ các kết quả đạt được, nhóm nhận thấy rằng việc kết hợp các phương pháp truyền thống với các kỹ thuật hiện đại như học sâu có thể đem lại hiệu quả tối ưu trong các hệ thống thông tin vô tuyến. Hướng phát triển tiếp theo có thể bao gồm:

\begin{itemize}
    \item Nghiên cứu các phương pháp kết hợp giữa học sâu và truyền thống để cải thiện hiệu suất ước lượng kênh.
    \item Tối ưu hóa mô hình học sâu để giảm độ phức tạp tính toán và tăng tốc quá trình huấn luyện.
    \item Mở rộng mô hình học sâu cho các hệ thống MIMO và 5G với nhiều tín hiệu đa dạng.
    \item Nghiên cứu các phương pháp học sâu khác như RNN, LSTM, GAN để so sánh hiệu suất với mô hình CNN.
    \item Tích hợp các phương pháp ước lượng kênh vào các hệ thống truyền thông thực tế như 5G, IoT, v.v.
\end{itemize}

Bài tập lớn không chỉ cung cấp những kiến thức hữu ích về ước lượng kênh mà còn gợi mở nhiều hướng nghiên cứu tiếp tục trong tương lai.