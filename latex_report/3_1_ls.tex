\subsection{Ước lượng kênh bằng LS}

Phương pháp ước lượng Least Squares (LS) là một trong những phương pháp đơn giản và phổ biến nhất để ước lượng kênh trong hệ thống thông tin vô tuyến. 
Mặc dù độ chính xác của nó không cao bằng các phương pháp khác như MMSE, nhưng LS có ưu điểm về tính đơn giản và không yêu cầu thông tin thống kê của kênh, điều này làm cho nó trở nên hiệu quả trong các ứng dụng không đòi hỏi hiệu suất tối ưu hoặc có giới hạn về tài nguyên tính toán.

\subsubsection{Nguyên lý của phương pháp LS}

Ý tưởng cơ bản của phương pháp LS là tìm ra ma trận kênh $\bm{H}$ 
sao cho sai số giữa tín hiệu nhận được $\bm{Y}$ và tín hiệu truyền đi $\bm{X}$, 
sau khi đi qua kênh, là nhỏ nhất. 
Phương pháp này dựa trên việc tối thiểu hóa sai số bình phương giữa tín hiệu thu nhận được và tín hiệu dự đoán (dựa trên mô hình kênh truyền và các ký tự pilot đã biết).

Giả sử mô hình kênh truyền được biểu diễn dưới dạng:

\begin{equation}
    \bm{Y} = \bm{H} \cdot \bm{X} + \bm{N}
\end{equation}

Trong đó:
\begin{itemize}
    \item $\bm{Y}$: Tín hiệu nhận được tại thiết bị thu.
    \item $\bm{X}$: Tín hiệu đã truyền đi từ máy phát.
    \item $\bm{H}$: Ma trận kênh vô tuyến biểu diễn đặc tính của kênh (suy hao, trễ, tán xạ).
    \item $\bm{N}$: Nhiễu trắng Gaussian cộng (AWGN), mô tả nhiễu ảnh hưởng đến tín hiệu.
\end{itemize}

Phương pháp LS tìm H sao cho sai số bình phương giữa $\bm{Y}$ và 
$\bm{H} \cdot \bm{X}$ là nhỏ nhất, tức là:

\begin{equation}
    \bm{H_{LS}} = \underset{\bm{H}}{\arg \min} \| \bm{Y} - \bm{H} \cdot \bm{X} \|^2
\end{equation}

\subsubsection{Công thức ước lượng LS}

Để giải bài toán tối thiểu hóa sai số bình phương, ta sẽ tính đạo hàm của hàm mục tiêu đối với $\bm{H}$ và đặt đạo hàm này bằng 0. Kết quả ta có được công thức ước lượng LS như sau:

\begin{equation}
    \bm{H_{LS}} = \bm{Y} \cdot \bm{X}^{+}
\end{equation}

Trong đó: \( \bm{X}^{+} \) là \textbf{nghịch đảo giả Moore-Penrose} của ma trận \( \bm{X} \). 
Nghịch đảo giả \( \bm{X}^{+} \) thường được sử dụng khi \( \bm{X} \) không phải là ma trận vuông hoặc không khả nghịch.

Nếu $\bm{X}$ là ma trận vuông và khả nghịch, công thức này đơn giản hơn:

\begin{equation}
    \bm{H}_{LS} = \bm{Y} \cdot \bm{X}^{-1}
\end{equation}

\subsubsection{Ước lượng LS tại các vị trí pilot}

Trong thực tế, tín hiệu pilot \( \bm{X_p} \) được chèn vào một số vị trí đã biết trong khung thời gian-tần số để hỗ trợ ước lượng kênh. 
Tại các vị trí có ký tự pilot, tín hiệu nhận được có thể được biểu diễn dưới dạng:

\begin{equation}
    \bm{Y_p} = \bm{H_p} \cdot \bm{X_p} + \bm{N_p}
\end{equation}

Trong đó:
\begin{itemize}
    \item \( Y_p \) là tín hiệu thu được tại các vị trí pilot,
    \item \( X_p \) là tín hiệu pilot đã truyền đi (đã biết),
    \item \( H_p \) là ma trận kênh tại các vị trí pilot cần được ước lượng,
    \item \( N_p \) là nhiễu tại các vị trí pilot.
\end{itemize}

Dựa trên công thức LS, ma trận kênh tại các vị trí pilot có thể được ước lượng như sau:

\begin{equation}
    \bm{H_{LS}} = \frac{\bm{Y_p}}{\bm{X_p}} = \bm{Y_p} \cdot \bm{X_p}^{+}
\end{equation}

\subsubsection{Nội suy để ước lượng kênh tại các vị trí không phải pilot}

Phương pháp LS chỉ ước lượng được ma trận kênh \( \bm{H} \) tại các vị trí có ký tự pilot. 
Để tính toán giá trị kênh tại các vị trí không có ký tự pilot, các phương pháp \textbf{nội suy hai chiều} (2D interpolation) sẽ được sử dụng. 
Một số phương pháp nội suy phổ biến bao gồm:

\begin{itemize}
    \item Nội suy tuyến tính (Linear interpolation),
    \item Nội suy spline,
    \item Nội suy bậc cao (Higher-order interpolation).
\end{itemize}

Nội suy giúp tạo ra một ma trận kênh ước lượng đầy đủ tại tất cả các vị trí trong khung thời gian và tần số, dựa trên thông tin từ các vị trí pilot.