\subsection{Ước lượng kênh bằng kĩ thuật học sâu}

Phương pháp ước lượng kênh truyền vô tuyến dựa trên kỹ thuật học sâu (Deep Learning - DL) đã nổi lên như một giải pháp tiên tiến trong các hệ thống thông tin không dây, 
đặc biệt trong bối cảnh các hệ thống ngày càng phức tạp như MIMO hay OFDM. 
Các kỹ thuật học sâu có khả năng tự động học mô hình và xử lý các vấn đề phi tuyến mà các phương pháp truyền thống như LS hay MMSE gặp khó khăn \cite{Soltani2019}. 

\subsubsection{Nguyên lý của ước lượng kênh dựa trên học sâu}

Phương pháp ước lượng kênh truyền dựa trên học sâu sử dụng mạng nơ-ron nhân tạo (Artificial Neural Networks - ANN) để học cách biểu diễn và dự đoán kênh truyền từ dữ liệu nhận được, 
thay vì dựa trên các mô hình thống kê hoặc các công thức toán học cố định như trong MMSE hoặc LS. 
Mục tiêu chính là học mối quan hệ giữa tín hiệu nhận được và kênh truyền, từ đó ước lượng đặc tính kênh một cách chính xác và tự động.

Kênh truyền có thể được coi như một bản đồ hai chiều (2D) trong miền thời gian và tần số. 
Ý tưởng của DL là mô hình hóa đáp ứng kênh truyền theo thời gian và tần số như một hình ảnh hai chiều và sử dụng các mạng nơ-ron tích chập để học cách ước lượng kênh từ dữ liệu đầu vào.

\subsubsection{Kiến trúc mạng học sâu cho ước lượng kênh}

Phương pháp học sâu cho ước lượng kênh có thể sử dụng nhiều loại kiến trúc khác nhau, nhưng phổ biến nhất là Mạng nơ-ron tích chập (CNN - Convolutional Neural Networks). 
CNN đặc biệt phù hợp cho việc xử lý dữ liệu dạng lưới như hình ảnh hoặc dữ liệu thời gian-tần số trong các hệ thống thông tin không dây.

\begin{table}[H]
    \centering
    \begin{tabular}{cccc}
        \hline
        \textbf{Thứ tự} & \textbf{Lớp} & \textbf{Kích thước ngõ ra} & \textbf{Số lượng tham số} \\
        \hline
        1 & Input & (số mẫu, 72, 14, 1) & 0 \\
        2 & Conv2D & (số mẫu, 72, 14, 64) & 5248 \\
        3 & Conv2D & (số mẫu, 72, 14, 32) & 2080 \\
        4 & Conv2D & (số mẫu, 72, 14, 1) & 801
    \end{tabular}
\end{table}

\subsubsection{Quy trình ước lượng kênh dựa trên học sâu}

Các bước chính trong quy trình ước lượng kênh dựa trên học sâu:

\textbf{Thu thập dữ liệu và tiền xử lý}

\begin{itemize}
    \item \textbf{Thu thập dữ liệu}: 
    Dữ liệu bao gồm các tín hiệu nhận được tại máy thu cùng với các thông tin pilot đã biết. 
    Các mẫu dữ liệu này sẽ được dùng để huấn luyện mô hình DL.
    \item \textbf{Chuẩn hóa dữ liệu}: 
    Dữ liệu phải được chuẩn hóa để phù hợp với đầu vào của mạng nơ-ron. 
    Ví dụ, dữ liệu có thể được chuẩn hóa về các phạm vi giá trị chuẩn để mô hình học dễ dàng hơn.
\end{itemize}

\textbf{Huấn luyện mô hình học sâu}

Mạng học sâu được huấn luyện trên dữ liệu từ kênh truyền để học cách ước lượng các giá trị kênh dựa trên các mẫu tín hiệu nhận được. 
Quá trình huấn luyện thường bao gồm:

\begin{itemize}
    \item \textbf{Xác định hàm mất mát}: Hàm mất mát thường được sử dụng là Mean Square Error (MSE) giữa kênh ước lượng và kênh thực, tức là:

    \begin{equation}
        \mathcal{L} = \frac{1}{N} \sum_{i=1}^{N} \| \bm{H_i} - \bm{\hat{H}_i} \|^2
    \end{equation}
    
    Trong đó \( \bm{H_i} \) là kênh thực và \( \bm{\hat{H}_i} \) là kênh ước lượng từ mạng nơ-ron.

    \item \textbf{Tối ưu hóa}: Để huấn luyện mạng nơ-ron, ta cần một tập dữ liệu lớn chứa các mẫu tín hiệu nhận được và kênh truyền tương ứng. Thuật toán tối ưu hóa được chọn là Adam với hàm mất mát là Mean Square Error (MSE) giữa kênh ước lượng và kênh thực. Tốc độ học (learning rate) được chọn là 0.001.
\end{itemize}

\textbf{Kiểm thử và đánh giá}

Sau khi huấn luyện, mô hình DL được kiểm thử trên một tập dữ liệu mới để đánh giá khả năng ước lượng kênh. 
Kết quả thường được so sánh với các phương pháp truyền thống như LS hoặc MMSE dựa trên các chỉ số như MSE hoặc Signal-to-Noise Ratio (SNR).